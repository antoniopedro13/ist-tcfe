\section{Theoretical Analysis}
\label{sec:theoretical}

\par In this section, the circuit shown in the previous figure is analysed theoretically.
\par Our approach begins with getting a smaller value for the voltage input in our circuit. As such, we used a transformer, with a ratio of turns in the primary and secondary circuits of $frac{1}{11}$. As a consequence, because the input voltage amplitude is $V_s=230V$, this value will be lowered to $V_r=frac{230}{11}V$, so that the circuit can then approximate to 12V. Although we already have a better value for the voltage input, it still is an AC input that needs to be converted to DC. In order to do this, we proceeded as described next.
\par Firstly, we used a full wave rectifier. As we said before, this circuit allows us to use the full length of the wave, converting the initial AC signal in an equal amplitude unidirectional current. Therefore, the wave format remains unchanged if the voltage is positive and gets reflected if the voltage is negative.
\par Secondly, we used a parallel of a capacitor and a resistor, in order to reduce the variaton of the voltage output, and make it closer to a DC one. This circuit is called an envelope detector, and it works based on the charge and discharge proccesses of the capacitor - when the diodes of the full-wave rectifier are on, the capacitor charges up, and when they turn off, the capacitor discharges through the resistor (in order to make things more clear, we're calling $v_O(env)$ to the voltage drop in $R_1$ terminals).
\par In order to compute the output of this process, we needed to know when the capacitor starts and stops discharging ($t_{off}$ and $t_{on}$, respectively). As one might guess, and based on what has been said, $t_{off}$ corresponds to the instant in which the diodes become off ($t_{off}=frac{1}{\omega} \cdot atan(frac{1}{\omega R_1C})$) and if $t<t_{off}$, $v_O(env)=v_R$. On the other hand, $t_{on}$ corresponds to the instant in which the diodes become on ($t{on}$ is obtained by solving the equation $Asin(\omega t{on})=Asin(\omega t_{off}) \cdot e^{-frac{t_{on}-t_{off}}{R_1C}}$) and if $t<t_{on}$, $v_O(env)=Asin(\omega t_{off}) \cdot e^{-frac{t-t_{off}}{R_1C}}$. The ripple voltage is given by max($v_O$(env)-min($v_O(env)$).

%tabela ripple envelope e average envelope

\par Lastly, we used a series of 20 diodes, in series with a resistor (voltage regulator circuit) in order to "get rid" of the AC component and make the DC component 12V. This makes the current an almost perfect 12V DC, because the diodes have a constant voltage output, no matter what the current or voltage inputs are. If we calculate $v_O(env)$ average from the last point ($V_O$), we can conclude if it sufices to turn the diodes on (this happens when $v_O(env) \geq nv_{ON}$, being $v_{ON}$ the minimum voltage required to activate one diode).
\par At this point, we have the DC component of the voltage, so we need to calculate and minimize the AC component. This is achieved by making incremental analysis. In incremental analysis, we can replace each diode by a resistor with resistance $r_d$, and we get $v_o = \frac{n \cdot r_d}{n \cdot r_d+R_2} \cdot (v_O(env)-V_O(env))$. As we can see, if $R_2 \gg n \cdot r_d$, $v_o \approx 0$.
\par Putting all this things together, we get a final output voltage $v_O=V_O+v_o$, and the average of the signal must be approximatly 12V.

%tabela ripple regulator e average regulator

%Não faria mais sentido se a condição de ligar os diodos fosse o minimo de v_O ser maior ou igual à tensão mínima requerida pelos diodos?
%Nos terminais de R1, a tensão média corresponde à componente DC, certo?

%graficos


	


