\section{Conclusion}
\label{sec:conclusion}

\par In this laboratory assignment the goal of determining




\par In this laboratory assignment the goal of determining node voltages as well as branch currents has been achieved. Both of the analyses have been performed from the values given by the \emph{Python} program - theoretically using the \emph{Octave} maths tool and by circuit simulation using the \emph{Ngspice} tool. As we mentioned in subsection~\ref{ssec:Error analysis}, the error is very small in branch current values (even zero in some cases) due to lower precision used by \emph{Ngspice}, and it assumes the value zero for all the node voltage values due to the fact that both \emph{Octave} and \emph{Ngspice} used the same precision. Taking this into account, we can safely say that the simulation results matched the theoretical results pretty well. The reason for this match is the fact that this is a circuit containing only linear components, so the theoretical and simulation models cannot differ. Although the match is almost perfect, we have to say that if the data were obtained in a real situation, this result would be very hard to achieve, since we had to take into acount other systhematic and accidental errors that might have occurred, as well as phenomena like joule effect in the conductors.  

