\section{Conclusion}
\label{sec:conclusion}


\par In this laboratory assignment, the goal was to create a circuit with the fuction of a signal amplifier, reaching the looked after gain for the central frequency of it whilst minimizing its cost. To do that theoretical and experimental analysis were conducted and the results studied.

\par Even though the model used in the octave script is a good approximation of a real Op amplifier, as the results between ngspice and octave differ  we can conclude that the model used isn't perfect. This may be explained because of the approximations used in the amplifier model (no current in the inputs of the amplifier and less the variables controlling it on the theroretical model when compared to the ngspice model), taking into account that these are non linear components. 
 
\par In conclusion, with this laboratory assignment we were able to understand more deeply the functioning of Op-Amps and its applicability on different devices. Remembering the goal of reaching the needed gain for the given frequency, we can say we are pretty satisfied with our results as we were able to achieve these goals and also improve on the merit obtained.







